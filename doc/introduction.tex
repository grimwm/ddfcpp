\mychapter{Introduction}

% Basic Goals
Operating systems are one of the core components to a computer.
In fact, operating system research has dominated much of the
early history of computer science, with much of the research
involving monolithic kernels that are used in operating systems
such as GNU/Linux or BSD.  However, a newer breed of kernels known
as microkernels delivers a new set of promises and problems.

Microkernels involve multiple design paradigms, including principles from
computer architecture, operating systems, and distributed systems.
They have always offered the promise of improved extensibility,
flexibility, reliability, and others; however, the largest
failure in early microkernel designs was performance.  To counteract
this failure, years of research have considerably improved
microkernel performance to the point where microkernels are now a
viable alternative to monolithic kernels.

Even though microkernel performance now roughly matches monolithic
kernel performance on most levels, one lingering problem with
microkernels is keeping device
drivers inside a protected memory region known as user-space without
sacrificing performance.  Analyses on recent work in these types of
device driver frameworks indicates that poor performance
is the result of poor design.  However, literature on
distributed systems indicates that even with a good design, the
overall problem is inherently difficult because it involves
communication amongst many nodes, creating a fully-connected graph of
communication links~\cite{cormen2001, tanenbaum2002}.
Therefore, the central issue is to find an optimal compromise between
functional perfection and responsiveness.

When a compromise between functional perfection and responsiveness
has been reached for a device driver framework, its design must be implemented
and analyzed.  Such analyses should include mathematical proofs to
demonstrate the design's correctness and its expected performance.
After a full implementation of the framework, empirical analyses on it
should be performed to demonstrate that the framework works in a practical
manner.  The empirical analyses should also compare the device driver
framework's performance against frameworks used in existing operating
systems as a benchmark.

A background of monolithic kernels and microkernels will be covered, with the
primary focus on microkernel performance and issues surrounding them.  Next,
specific examples of existing microkernels will be given, along with research
related to device driver frameworks in microkernel operating systems.
Finally, the design of this thesis' device driver framework will be discussed,
along with the results obtained from it, possible future directions it can
take, and concluding information.
