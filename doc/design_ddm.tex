\mysection{Device Driver Manager}

The device driver manager is a server task that manages
drivers and allows other tasks to find and gain access to them.
Due to the fact that the DDM is the core process in the DDF, the DDM is
resilient to failures of devices or their drivers.

\mysubsection{Bootstrap Protocol}

Bootstrapping is a recurring theme with microkernel-based operating systems;
however, it is doubly difficult with a device driver framework.
A true microkernel is the first program
loaded, knows nothing about devices, and must rely on external drivers to do
I/O.  However, since a true microkernel does not know enough about the
hardware to load drivers off of secondary storage, it becomes useless.

Luckily, Minix 3 comes with some utilities that make it easy to create a
bootable image consisting of the microkernel and some processes which can
include the DDM and a few drivers.  This entirely eliminated the need for
me to implement this myself, but it is quite possible that porting this DDF
to another platform will reinstate the need to write this protocol.

\mysubsection{Fault Tolerance}

Since all drivers reside in their own address space and are assumed to
reside on the same system, the DDM does not need extensive fault tolerance.
When a process unexpectedly dies and another process tries to communicate
with it, the process manager will tell the calling process why the call
fails.  Likewise, when a failure is recognized, a log message is generated,
and it is up to the administrator to resolve the problem.  However, since
the existing implementation of this design is on Minix 3, Minix's
reincarnation server (RS) can reincarnate processes that have died
unexpectedly.
