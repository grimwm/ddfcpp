\mysection{The Microkernel}

In contrast to monolithic kernels, which place operating system specific code
directly in kernel-space, pure microkernels put only essential operating
system functions into
kernel-space; all other services are placed outside the kernel.  Microkernels
replace the vertical scaling of monolithic kernels with horizontal
scaling~\cite{stallings2005}, as shown in \figurename~\ref{fig:kernel_design}.
An example of how microkernels represent a client/server architecture is
shown in \figurename~\ref{fig:microkernel_servers}.

	\begin{figure}[tb]
	\begin{center}
	\scale{0.50}{\graphic{kernel_design}}
	\end{center}
	\caption{Kernel design in (a) monolithic kernels and (b)
	microkernels~\cite{stallings2005}.}
	\label{fig:kernel_design}
	\end{figure}

	\begin{figure}[tb]
	\begin{center}
	\psfrag{m$_0$}{m$_0$}
	\psfrag{m$_1$}{m$_1$}
	\psfrag{m$_2$}{m$_2$}
	\psfrag{m$_3$}{m$_3$}
	\scale{0.50}{\graphic{microkernel_servers}}
	\end{center}
	\caption{Microkernels exhibit client/server behaviour.}
	\label{fig:microkernel_servers}
	\end{figure}

The microkernel's unique design pattern maintains several advantages over
monolithic kernels~\cite{stallings2005}:

\begin{itemize} % Microkernel Advantages

\item {\important Uniform Interface}:
Message passing provides a consistent method for accessing services
on remote servers.  This means that multiple services on one or many
computers could interact with each other using the same message
passing mechanism.  Further, depending on particular
design decisions, message passing can allow multiple computers
to communicate with varying degrees of architecture
independence.

\item {\important Extensibility}: Services can be added or extended
relatively easily without rebuilding or rebooting the operating system
due to the fact that all services run independently of each
other.  This is also partly because the
loosely-coupled code has the tendency to be smaller, making it easier to
change without needing to review as much code.

\item {\important Flexibility}: Microkernel operating systems can be easily
stripped down or modified to meet changing demands.
For example, one packaging of the operating system may need to handle
high-availability (HA) and high-performance computing (HPC), while another
packaging may need to run on a PDA or game console.  Meeting these changing
demands in a monolithic operating system would take more development time
than a microkernel operating system.

\item {\important Reliability}: All code can be rigorously tested, since
the code base is small.  Some areas where reliability
can be important are NASA spacecraft, medical systems, banking,
real-time environment monitoring, and others.

\item {\important Portability}: Architecture-dependent code only
exists in the microkernel and special user-space servers, leaving
the majority of the software architecture-independent.

\item {\important Inherent Distributed System}: If all operating
system tasks have globally unique identifiers, then in effect, the
operating system is a single system image (SSI) at the
microkernel level.  This is due to the fact that
the microkernel provides the message passing mechanism and basic
naming system for tasks.  A
simple example of this fact is that the process scheduling server will have
to interact, via messages, with the virtual memory server as it swaps
processes in and out of the CPU and memory.

\item {\important Object-Oriented / Component-Oriented Design}:
{\important Object-oriented} methodologies dictate the tight coupling of
related data and methods into single entities known as
{\important objects}; likewise,
{\important component-oriented} methodologies dictate the tight coupling
of related objects into single entities known as
{\important components}~\cite{tanenbaum2002}.
This is a natural evolution of operating system design as a result of the
microkernel's inherently distributed operation.
Both of these methodologies work together seemlessly,
providing a clear context of what everything in the operating system does.

\end{itemize} % Microkernel Advantages

All the features just mentioned interact to allow microkernel operating
systems to be easily extended into several application domains, such as
hard and soft real-time systems, high-availability systems, high
performance systems, or any combination of these and
others.

As operating system history, and in general, software development history
has unfolded, it has become clear that object-oriented approaches to
large-scale problems are often beneficial to reducing the complexity of a
given system.  In fact, operating system researchers once theorized that
a similar approach to kernel design could alleviate the shortcomings of
the monolithic kernel by splitting primary and secondary services into
two groups.  The set of primary services, such as IPC and kernel thread
scheduling would remain in kernel-space with the
microkernel, and everything else would be placed in user-space, where
memory addressing is
protected\footnote{Memory is securely partitioned so that one unprivileged
task cannot access another task's memory via arbitrary addresses.  Any
requests by an external task to receive another task's memory must be
explicitly granted by a memory manager.}.

\mysection{Microkernel Performance}

Microkernels have a history of performing poorly, because it takes time to
build messages, send them from one end, receive them at another end,
and decode them.  However, the amount of performance
penalty is difficult to analyze and depends upon particular implementations
of message passing~\cite{stallings2005}.

In fact, first generation microkernels, such as Mach and Chorus,
often suffered large performance penalties due to complex message passing
systems and large, relatively complex code bases.  Even
optimized versions of these microkernels still exhibited the performance
problems~\cite{stallings2005}.

Surprisingly, after intense examination of first generation microkernel
performance, it was realized that the overhead for context switching was
not the problem, even though initial intuition suggests a larger
number of context switches would be the cause.  In fact, the main problem
lay in the inherent IPC overhead in first generation microkernels every
time a remote procedure call
(RPC)\footnote{In this context, this is synonymous to doing a system
call in a monolithic system.  In actuality, however, RPCs allow a
client to remotely execute code at a remote location and get results
back, all in a transparent fashion.}
was executed.  As an example, optimized first
generation microkernels execute RPCs roughly eight times {\important slower}
than their Unix system call counterparts~\cite{liedtke1996}.


To mitigate the performance problems that first generation microkernels were
suffering, two branches of development took place.  The first branch decided
to keep the same generation of microkernels but move some critical servers
and device drivers back into the kernel, based on empirical observations of
good performance in monolithic operating systems; this led to a breed of
kernels known as hybrid kernels.
The other branch of microkernel development led to a more radical
redesign of microkernels, aiming to decrease the size of the microkernel
as much as possible, thus eliminating enough interprocess communication
overhead to make them practical for general use; this branch of
development has given way to the second generation of
microkernels~\cite{liedtke1996}.  A comparison of the code size between a
first and second generation microkernel is shown in
\tablename~\ref{tab:microkernel_comparison}.

Both hybrid kernels and second generation microkernels had tradeoffs.  In
hybrid kernels, the tradeoffs included better performance, a larger, less
flexible kernel, and more interfaces rather than fewer.  In contrast,
second generation microkernels completely eliminate performance issues;
however, it remains an open question as to whether second generation
microkernels primitives are flexible enough for modern
demands~\cite{stallings2005}.

	\begin{table}[tb]
	\begin{center}
	\begin{tabular}{|c|c|c|c|c|}
	  \toprule
	  \bf Microkernel & \bf Version & \bf Size (KB) &
	  \bf System Calls & \bf Generation\\
	  \midrule
	  GNU Mach & 1.3 & $\approx$300 & 222 & 1\\
	  L4Ka::Pistachio & X.2 & $\approx$12 & 12 & 2\\
	  \bottomrule
	\end{tabular}
	\end{center}
	\caption{Microkernel comparison~\cite{mach_ref_2001,l4ref2005}.}
	\label{tab:microkernel_comparison}
	\end{table}


\mychapter{Bootstrapping Microkernel Systems}

Pure microkernels introduce problems when trying to boot a computer, also known
as {\important bootstrapping}.  For example, when the microkernel first
loads onto the system, it has no working knowledge of devices and file
systems; however, in order for the microkernel to get any
work accomplished, by conventional standards, it needs to
access file systems on hardware devices in order to load more software.  This
introduces a chicken-and-the-egg problem
whereby the kernel needs to know about devices whose device drivers are
separate from the kernel; therefore, something else must
be done, such as using external boot-time modules that provide the
necessary information.

In general, another problem with bootstrapping is that historically, boot
loaders have been written for specific operating systems.  These
boot loaders are loaded by the Basic Input/Output System (BIOS) and
are used to boot the rest of the kernel.  However, the
Multiboot specifiction and one implementation of it helps solve this
problem and others on the Intel IA-32 architecture~\cite{grub_info}.

\mysubsection{Intel 32-bit Architecture}

Before getting into the details of bootstrapping, a bit of background on
Intel's 32-bit architecture (IA-32) is required.  This information is
necessary because it describes the difference between little endian
and big endian.  When working with binary data at the hardware level,
this issue becomes important since passing data in a big endian format
when the hardware expects it in a little endian format will confuse the
hardware.  This section also introduces the concept of linear address
spaces and is important to understand how it abstracts memory addressing
at a higher level.

Intel's byte addressing is {\important little endian}~\cite{ia32-1-2004},
meaning each
{\important word}'s\footnote{32-bit memory segments~\cite{hennessy2003}.}
least-significant-byte lays on the right-hand side
and most-significant-byte lays on the left-hand side.
For an example of the difference between little endian and big endian
byte ordering, see \figurename~\ref{fig:endianness}.

	\begin{figure}[tb]
	\begin{center}
	\mygraphic{endianness.eps}
	\end{center}
	\caption{Endianness for half words: (a) big endian and (b) little
		endian.}%~\cite{ia32-1-2004}.}   XXX
	\label{fig:endianness}
	\end{figure}

All memory addresses in IA-32 are
{\important byte addressable}~\cite{ia32-1-2004}.  What
this means is that all 32-bits of an IA-32 memory address can be used to
access up to 4 GB of memory, as expressed in Equation~\ref{eq:max_mem}.

\begin{equation}
2^{32} B = 4,294,967,296 B *
\frac{1~KB}{1,024~B} * \frac{1~MB}{1,024~KB} * \frac{1~GB}{1,024~MB} =
4 GB
\label{eq:max_mem}
\end{equation}

{\important Linear address space} is a typical hardware phrase indicating the
fact that some sort of memory mapping hardware is available to convert a
given 1-dimensional address into an N-dimensional cross-section in
real memory~\cite{hennessy2003}.  Consider the 32-bit address, 0x1000FF98.
When the memory mapping hardware receives this address, one possible way
to do the mapping is by using a row address
strobe\footnote{Also known as RAS, the address is used to find a
particular row in memory matching that address~\cite{hennessy2003}.}
followed by a column address
strobe\footnote{Also known as CAS, it works almost like RAS, except it
works on the row found in RAS to find a particular column~\cite{hennessy2003}.}
to return data to the requester~\cite{hennessy2003}.  Under this
method, the mapping hardware will use 0x1000 to find a row in real memory
and 0xFF98 to find a particular column (i.e. bit) in that
row~\cite{hennessy2003}.  The important thing to remember is that all
memory appears to be logically ordered as a linear set of addresses, while
in reality the memory mapping hardware could map the addresses to a foreign
set of rules.

\mysubsection{Multiboot Specification}
To deal with the problem of booting microkernel-based operating systems and
GNU Hurd in particular, Erich Boleyn and Brian Ford devised the Multiboot
specification~\cite{grub_info}.  Essentially, the Multiboot specification
solves two core problems.  The first, more classic problem it solves is a
consistent bootstrapping protocol for various types of kernels
(Linux, BSD, etc.).  The second, more important problem it solves for
microkernels is the ability to load a kernel and several kernel servers
(modules) at boot time~\cite{multiboot_info}.

To solve the bootstrapping protocol problem, the Multiboot specification
requires that kernel images contain a special header that
gives information about the file layout.  This header is known as the
{\code Multiboot header} and is shown in
\figurename~\ref{fig:multiboot_header}.  The Multiboot header is a consistent
data structure that can be prepended to kernel images to create kernels that
Multiboot-compliant boot loaders can understand.

	\begin{figure}[tb]
	\begin{center}
	\mygraphic{multiboot_header.eps}
	\end{center}
	\caption{Multiboot header fields: (a) always required, (b) required if
	{\code flags[16]} set, and (c) required if {\code flags[2]} set.}
	\label{fig:multiboot_header}
	\end{figure}

The Multiboot header in \figurename~\ref{fig:multiboot_header} is also known as
a magic header and eliminates the
need for kernel developers to write special purpose boot loaders that
are only compatible with specific kernels by specifying three critical
fields: {\important magic}, {\important flags}, and
{\important checksum}~\cite{multiboot_info}.  The
{\important magic}\footnote{Magic values provide information about the rest of
the program, such as its linking type.}
field is located at relative
address\footnote{If a program located at physical memory location 0xAE98
tries to access relative memory location 0x0040, it will actually
access physical memory address \( 0xAE98+0x0040=0xAED8 \).}
$0x0$ and defined to be $0x1BADB002$, while {\important flags} specifies
features the operating system requests or requires of the boot loader.  Also,
{\important flags} indicates to the boot loader that the Multiboot
header contains optional information that should be used~\cite{multiboot_info}.
Finally, when the values in {\important magic} and {\important flags}
are summed with {\important checksum}, the result should be a
32-bit unsigned zero~\cite{multiboot_info}.

For implementors of microkernel-like kernels, the Multiboot specification
solves an even greater problem: the loading of several software modules at
boot time.  This is a critical component to writing microkernel operating
systems, because it allows developers to write independent code segments,
apart from the microkernel, which are necessary to facilitate the full
operation of operating systems.  Once a kernel and its modules are
in memory, the bootloader notifies the Multiboot-compliant kernel of the
modules and their locations in memory via the Multiboot information
structure\footnote{On IA-32, the Multiboot-compliant kernel knows where
the Multiboot information structure is via the physical address set
in the EBX register.}
seen in \figurename~\ref{fig:multiboot_info}~\cite{multiboot_info}.

	\begin{figure}[tb]
	\begin{center}
	\mygraphic{multiboot_info.eps}
	\end{center}
	\caption{(a) Multiboot information structure and (b) array of
	Multiboot module structure(s) that exists if
	{\code flags[3]} set and {\code mods\_count} $\geq~0$.}
	\label{fig:multiboot_info}
	\end{figure}

\mysubsection{{GNU GRUB}}
GNU GRUB (GRUB) is the first boot loader ever created with the Multiboot
specification in mind~\cite{grub_info}.
It was conceived by Erich Boleyn in 1995 while he was
attempting to modify FreeBSD's boot loader to conform to the Multiboot
specification; he was doing this in an attempt to get GNU Hurd to boot
using University of Utah's Mach 4 microkernel.  However, modifying
FreeBSD's boot loader proved to be an unreasonble amount of work, and Erich
decided to begin GRUB.  Over time, Erich's priorities changed, and GRUB
became a GNU project in 1999, being adopted by Gordon Matzigkeit and
Yoshinori Okuji~\cite{grub_info}.

Currently, GRUB only runs on the IA-32 architecture and is primarily
aimed at supporting the GNU Hurd operating system; though, it is
likely GRUB will be ported to other architectures in the future (especially
when the Hurd extends beyond IA-32).
\figurename~\ref{fig:grub_features}~\cite{grub_info} gives a list of GRUB's
major features.

	\begin{figure}[tb]
	  \begin{center}
	  \begin{minipage}{6in}
	  \begin{enumerate}

	  \item Implement full Multiboot specification compliance.

	  \item Achieve a straight-forward interface to end-users.

	  \item Employ rich interface for kernel experts.

	  \item Maintain backward boot compatibility for BSD, Linux, and
	    various proprietary kernels, such as Microsoft\copyright
	    Windows\texttrademark.

	  \item Recognize several executable formats, such as a.out variants
	    and the Executable and Linking Format (ELF).

	  \item Use human-readable configuration files with preset boot
	    information.

	  \item Provide menu of configuration presets when booting.

	  \item Provide a boot-time command-line shell for on-the-fly boot
	    customization before booting a kernel.

	  \item Support several common filesystem types to make GRUB
	    filesystem independent.

	  \item Support drive access from any drive recognized by the BIOS
	    and work independently of drive geometry translations.

	  \item Detect all installed RAM.

	  \item Support large disks using logical block address (LBA) mode.

	  \item Support booting kernels via network protocols (e.g. TFTP).

	  \item Support local access via remote terminals (e.g. serial line
	    access).
	  \end{enumerate}
	  \end{minipage}
	\end{center}
	\caption{Major list of GNU GRUB features.}
	\label{fig:grub_features}
	\end{figure}

The list of GRUB features is large, providing a powerful boot loader
implementation that is extremely useful to microkernels because of
its Multiboot compliance in loading kernel modules.  However, GRUB
does lack ability in two key areas: it is currently bound to the IA-32
architecture, and it can only load kernels at memory addresses
$\geq~1~MB$~\cite{grub_info}.

On modern PC systems, it is unlikely that any kernel would be less than 32-bit
addressable.  However, GRUB forces kernels to support at least 32-bit memory
addressing, because GRUB enables IA-32's 32-bit address mode and places
kernels at the 1 MB (0xFA000) or larger memory address~\cite{grub_info}.  The
reason for
this is because on IA-32, the linear address space is byte addressable,
meaning a 16-bit kernel can only reach 64 KB (0xFFFF) of main memory; thus, a
kernel
must boot into IA-32's protected mode at a minimum, which is a 32-bit mode,
giving the kernel direct access to 4 GB of linear address
space~\cite{ia32-1-2004}.  Fortunately, GRUB takes care of the special boot
code that places the CPU in 32-bit protected mode, greatly easing the kernel
implementor's job and freeing them from implementing specialized boot
loaders~\cite{grub_info}.

