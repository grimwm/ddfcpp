\mysection{L4Ka::Pistachio}

The L4Ka::Pistachio (L4Ka) kernel is an implementation of the original
L4 design by Jochen Liedtke~\cite{herder2005}.  L4Ka's main features are
simplicity, speed and security.

L4Ka has a straightforward application programming interface (API)
consisting of twelve system calls and twelve data types~\cite{l4ref2005}.
These system calls and data types handle the most
basic and central operation of microkernels: the kernel interface page
and API information, threading, timing and thread
scheduling\footnote{Does not refer to typical scheduling algorithms for
simulating parallel processing.  Instead, it allows threads to schedule a
thread of equal or lesser priority to run immediately.}, address
spaces and mapping, IPC, exception handling, and memory and processor
control~\cite{l4ref2005}.  All other services normally seen in monolithic
kernels are considered second class, to be handled by user-space
servers (threads).

Along with L4Ka's number of system calls and data types being small, its
memory footprint is as well, being only 12 KB.  Thus, L4Ka usually
resides in the CPU cache, and it exchanges most data via CPU
registers~\cite{l4ref2005}.  Both of these attributes are its primary source
of increased performance.
