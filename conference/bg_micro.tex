

%\mychapter{The Microkernel}
\mysubsection{The Microkernel}

In contrast to monolithic kernels, which place operating system specific code
directly in kernel-space, pure microkernels put only essential operating
system functions into
kernel-space; all other services are placed outside the kernel.  Microkernels
replace the vertical scaling of monolithic kernels with horizontal
scaling~\cite{stallings2005}, as shown in \figurename~\ref{fig:kernel_design}.
An example of how microkernels represent a client/server architecture is
shown in \figurename~\ref{fig:microkernel_servers}.


	\begin{figure}[tb]
	\begin{center}
	\psfrag{m$_0$}{m$_0$}
	\psfrag{m$_1$}{m$_1$}
	\psfrag{m$_2$}{m$_2$}
	\psfrag{m$_3$}{m$_3$}
	\mygraphic{microkernel_servers.eps}
	\end{center}
	\caption{Microkernels exhibit client/server behaviour.}
	\label{fig:microkernel_servers}
	\end{figure}

A microkernel operating system is a set of user-space servers built on top of
an existing microkernel~\cite{stallings2005}, such as Mach or L4, with one
critical service being the device driver framework.  The device driver
framework is the backbone service for all devices in the system, providing
a common way for device drivers to access devices and input/output (I/O)
buses on behalf of requesting clients.  
%
%The microkernel's unique design pattern maintains several advantages over
%monolithic kernels~\cite{stallings2005}:
%
%\begin{itemize} % Microkernel Advantages
%
%\item {\important Uniform Interface}:
%Message passing provides a consistent method for accessing services
%on remote servers.  This means that multiple services on one or many
%computers could interact with each other using the same message
%passing mechanism.  Further, depending on particular
%design decisions, message passing can allow multiple computers
%to communicate with varying degrees of architecture
%independence.
%
%\item {\important Extensibility}: Services can be added or extended
%relatively easily without rebuilding or rebooting the operating system
%due to the fact that all services run independently of each
%other.  This is also partly because the
%loosely-coupled code has the tendency to be smaller, making it easier to
%change without needing to review as much code.
%
%\item {\important Flexibility}: Microkernel operating systems can be easily
%stripped down or modified to meet changing demands.
%For example, one packaging of the operating system may need to handle
%high-availability (HA) and high-performance computing (HPC), while another
%packaging may need to run on a PDA or game console.  Meeting these changing
%demands in a monolithic operating system would take more development time
%than a microkernel operating system.
%
%\item {\important Reliability}: All code can be rigorously tested, since
%the code base is small.  Some areas where reliability
%can be important are NASA spacecraft, medical systems, banking,
%real-time environment monitoring, and others.
%
%\item {\important Portability}: Architecture-dependent code only
%exists in the microkernel and special user-space servers, leaving
%the majority of the software architecture-independent.
%
%\item {\important Inherent Distributed System}: If all operating
%system tasks have globally unique identifiers, then in effect, the
%operating system is a single system image (SSI) at the
%microkernel level.  This is due to the fact that
%the microkernel provides the message passing mechanism and basic
%naming system for tasks.  A
%simple example of this fact is that the process scheduling server will have
%to interact, via messages, with the virtual memory server as it swaps
%processes in and out of the CPU and memory.
%
%\item {\important Object-Oriented / Component-Oriented Design}:
%{\important Object-oriented} methodologies dictate the tight coupling of
%related data and methods into single entities known as
%{\important objects}; likewise,
%{\important component-oriented} methodologies dictate the tight coupling
%of related objects into single entities known as
%{\important components}~\cite{tanenbaum2002}.
%This is a natural evolution of operating system design as a result of the
%microkernel's inherently distributed operation.
%Both of these methodologies work together seemlessly,
%providing a clear context of what everything in the operating system does.
%
%\end{itemize} % Microkernel Advantages
%
This allows microkernel operating
systems to be easily extended into several application domains, such as
hard and soft real-time systems, high-availability systems, high
performance systems, or any combination of these and
others.

As operating system history, and in general, software development
history has unfolded, it has become clear that object-oriented
approaches to large-scale problems are often beneficial to reducing
the complexity of a given system.  In fact, operating system
researchers once theorized that a similar approach to kernel design
could alleviate the shortcomings of the monolithic kernel by splitting
primary and secondary services into two groups.  The set of primary
services, such as IPC and kernel thread scheduling would remain in
kernel-space with the microkernel, and everything else would be placed
in user-space, where memory addressing is protected. In this sense, a
microkernel approach can also be viewed as object-oriented in nature.

\mysection{Microkernel Performance}

Microkernels have a history of performing poorly, because it takes time to
build messages, send them from one end, receive them at another end,
and decode them.  However, the amount of performance
penalty is difficult to analyze and depends upon particular implementations
of message passing~\cite{stallings2005}.

In fact, first generation microkernels, such as Mach and Chorus,
often suffered large performance penalties due to complex message passing
systems and large, relatively complex code bases.  Even
optimized versions of these microkernels still exhibited the performance
problems~\cite{stallings2005}.

Surprisingly, after intense examination of first generation microkernel
performance, it was realized that the overhead for context switching was
not the problem, even though initial intuition suggests a larger
number of context switches would be the cause.  In fact, the main problem
lay in the inherent IPC overhead in first generation microkernels every
time a remote procedure call
(RPC)\footnote{In this context, this is synonymous to doing a system
call in a monolithic system.  In actuality, however, RPCs allow a
client to remotely execute code at a remote location and get results
back, all in a transparent fashion.}
was executed.  As an example, optimized first
generation microkernels execute RPCs roughly eight times {\important slower}
than their Unix system call counterparts~\cite{liedtke1996}.


To mitigate the performance problems that first generation microkernels were
suffering, two branches of development took place.  The first branch decided
to keep the same generation of microkernels but move some critical servers
and device drivers back into the kernel, based on empirical observations of
good performance in monolithic operating systems; this led to a breed of
kernels known as hybrid kernels.
The other branch of microkernel development led to a more radical
redesign of microkernels, aiming to decrease the size of the microkernel
as much as possible, thus eliminating enough interprocess communication
overhead to make them practical for general use; this branch of
development has given way to the second generation of
microkernels~\cite{liedtke1996}.  A comparison of the code size between a
first and second generation microkernel is shown in
\tablename~\ref{tab:microkernel_comparison}.

Both hybrid kernels and second generation microkernels had tradeoffs.  In
hybrid kernels, the tradeoffs included better performance, a larger, less
flexible kernel, and more interfaces rather than fewer.  In contrast,
second generation microkernels completely eliminate performance issues;
however, it remains an open question as to whether second generation
microkernels primitives are flexible enough for modern
demands~\cite{stallings2005}.

	\begin{table}[tb]
	\begin{center}
	\begin{tabular}{|c|c|c|c|c|}
	  \toprule
	  \bf Microkernel & \bf Version & \bf Size (KB) &
	  \bf System Calls & \bf Generation\\
	  \midrule
	  GNU Mach & 1.3 & $\approx$300 & 222 & 1\\
	  L4Ka::Pistachio & X.2 & $\approx$12 & 12 & 2\\
	  \bottomrule
	\end{tabular}
	\end{center}
	\caption{Microkernel comparison~\cite{mach_ref_2001,l4ref2005}.}
	\label{tab:microkernel_comparison}
	\end{table}

