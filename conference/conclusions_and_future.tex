\mychapter{Conclusions and Future Work}

In the end, the new user-space device driver framework is a success in
terms of functionality.  
%It has stepped the existing Minix framework into
%the foray of C++ driver development and has added a device driver manager
%that does not exist in Minix's current framework.
%
With the framework existing in user-space, safety of driver development
has improved.  When drivers ran in kernel-space, any flaws in the driver
could easily bring down an entire computer system by corrupting the operating
system kernel.  However, with drivers in user-space, failing drivers can
only impact processes directly relying on them.  This can still result in
an entire system grinding to a halt, but the possibility of this is lessened.

The new DDM this new framework provides is able to give
users a better presentation of the driver layer by bestowing users with a
hierarchical view of the driver tree rather than the flat view given by
Minix's process manager.

While this design has improved certain areas of driver development,
it still leaves room for enhancement.  Proceeding is a listing of potential
areas that could be enhanced or extended along with a description of how
the enhancement might be accomplished.

Adding DMA memory allocation to {\em libMem} would be a large step towards
making the new DDF more competitive with existing frameworks.  Currently,
memory allocation is not guaranteed to be contiguous in physical memory,
and this guarantee needs to be made for DMA-capable hardware and drivers.

%Another critical area for performance improvements could be allowing
%drivers full access to their request queue.  However, for this to be
%achievable on some operating systems, such as Minix 3, much work needs to be
%done to create a driver-accessible random access queue for drivers.

It would also be useful for the DDF's libraries to be shared objects that could
be dynamically linked into code using the DDF.  This would allow drivers to
take advantage of updated libraries as they become available without having
to rebuild the drivers.

%In order to allow the device driver manager to be upgraded without rebooting
%the system, an upgrade protocol is needed.  The new DDM could copy the
%state from the original DDM and then tell the process information task
%to direct new DDM requests to the new DDM.  The old DDM should terminate
%once the transfer of state information is complete, but the transfer should
%only happen when the old DDM is not busy.

To improve the fault tolerance of the DDM, the DDM should recognize
when a registered driver terminates unexpectedly.  It should then unregister
the driver and attempt to revive it, logging the problem that occurred.

Adding an access control layer to the DDF is an area that deserves attention.
The DDF as currently implemented provides ``implementation'', but another
layer needs to provide ``access''.  These areas should remain separate, and
one way to add the access control layer would be to integrate the DDF
with existing file systems.  This would have the advantages of being able
to use drivers as files while the file system controls access to these
special files.  However, for this to work, the framework would need to
have device classes the file system understand.

%Extending the idea of access controls, they would be needed as one part
%of security in a driver framework that supports remotely accessible
%drivers.  Again, the file system layer could be useful here by providing
%a simple way of accessing remote drivers and handling access control
%and other security issues.
