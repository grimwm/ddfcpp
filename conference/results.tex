\mychapter{Experimental Results}

In the user-space device driver framework, correctness was analyzed via
unit tests and mock objects, and performance was proven to be similar
to Minix's existing framework.  To analyze performance, a single working
driver in both our framework and Minix's existing framework was created to
generate timing measurements.

%\mychapter{Correctness}
\mysubsection{Correctness}

The unit testing framework is a custom set of classes that can be fed any type
of data.  The data fed to the unit tests is always
{\important (actual result, expected result)} tuples.
%  If the results do
%not match, a message is printed that explains the issue; otherwise, a message
%is printed indicating success.
Each unit test is considered valid when the two values match, while a failure is indicated when the two values differ.

%Unit tests are not only good at determining whether code executes correctly
%when correct inputs are given, but they excel at catching certain bugs in code
%that do not properly handle incorrect inputs.  For example, if code should
%return specific error codes on invalid input, unit tests can be used to check
%that the code does indeed return the correct error codes.  Further, it's
%entirely possible that the code being tested does not work at all and just
%crashes; unit tests are useful here for determining exactly when the
%crash occurs.
%
%Where unit tests fail is when output from code is non-deterministic; for these
%cases, mock objects pick up the slack.  Mock objects operate by inheriting a
%real object's traits and collecting statistical data about its use, such as
%the order of methods called, frequency of calls, and the length of each call;
%though, any or none of these particular data may be recorded, depending
%on the mock object developer's mood du jour.

For most of the DDF's code, unit tests were the obvious choice in proving teh correctness of the implemented framework.  One
exception to this was the keyboard, which produced non-deterministic output
since the data was entered onto a keyboard by a human.  In this case, a mock
object was used to record whether a method was called or not.  When the driver
was unloaded, a unit test was executed to validate that all methods were
called; this test always succeeded.

%\mychapter{Performance}
\mysubsection{Performance}

To ensure performance in the new DDF was similar to that of Minix's existing
DDF, timing measurements of several method calls were taken,
building information about method call overhead.  The methods themselves
did nothing and just returned, and these measurements were taken on four
typical methods: {\important open, read, write}, and {\important close}. 
% GNU
%Compiler Collection (GCC) version 4.1.1 was used to build tests with exception
%handling disabled and full optimizations enabled.  
The averaged results of
5,000 calls to each method is shown in \tablename~\ref{tab:results}.

	\begin{table}[tb]
	\begin{center}
	\begin{tabular}{|c|c|c|c|c|}
%	  \toprule
	\hline
	  \bf Framework & \bf Open & \bf Read & \bf Write & \bf Close\\
%	  \midrule
	\hline
	  Existing & 0.0042 & 0.0056 & 0.0050 &
	  0.0042 \\
%	  \midrule
	  New & 0.0054 & 0.0048 & 0.0042 &
	  0.0062 \\
	\hline
%	  \bottomrule
	\end{tabular}
	\end{center}
	\caption{DDF method call timing averages in ticks.}
	\label{tab:results}
	\end{table}

Averaging the timing results in \tablename~\ref{tab:results}
%\footnote{The fact
%that reads and writes were faster than opens and closes does not construe
%the fact that this always happens.  In this case, reads and writes just
%happened to execute faster.}
, one can see
that the existing framework averages 0.0048 clock ticks per call while our framework averages 0.0052 clock ticks per call.  Further, 
\tablename~\ref{tab:result_errors} shows that most of the calls took
no recorded clocks ticks to execute, and the rest of the calls only took
one clock tick to execute.  Thus, it can be seen that the averages are
based on numbers that do not vary greatly, providing strength to the
argument that the averages are indeed representative of general method
call overhead.

	\begin{table}[tb]
	\begin{center}
	\begin{tabular}{|c|c|c|c|c|}
%	  \toprule
	\hline
	  \bf Framework & \bf Open & \bf Read & \bf Write & \bf Close\\
%	  \midrule
	\hline
	  Existing (0) & 4,979 & 4,972 & 4,975 & 4,979\\
	  Existing (1) & 21 & 28 & 25 & 21\\
%	  \midrule
	\hline
	  New (0) & 4,973 & 4,976 & 4,979 & 4,969\\
	  New (1) & 27 & 24 & 21 & 31\\
%	  \bottomrule
	\hline
	\end{tabular}
	\end{center}
	\caption{Number of method calls that took 0 or 1 ticks to complete.}
	\label{tab:result_errors}
	\end{table}

Based on the results of \tablename~\ref{tab:results} and
\tablename~\ref{tab:result_errors}, it is shown that the new DDF
is roughly 6.8\% slower than the existing DDF in terms of call overhead,
proving that the new DDF's method call speed is similar to Minix's existing
device driver framework by less than an order-of-magnitude.  However, in
general operation, it is believed that method execution times would be much
larger than method call overhead, making this slightly degraded performance
even less noticeable.

%The most likely cause for the higher call overhead in the new DDF is the fact
%that C++ is used in the thesis' design.  C++ has additional overhead in
%object-oriented settings since it has to dereference objects to get access
%to its virtual function table.
%Typically, C++ would also have exception handling overhead, but that was
%disabled for theses tests since C++ system code likely would not use it.
%
%Having said that C++ is likely to blame for the method call degradation, it
%may actually improve overall execution times in a real setting.  It could
%do this because it provides more type information to the C++ compiler,
%allowing it to optimize more.
