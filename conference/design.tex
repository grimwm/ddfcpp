\mychapter{Device Driver Framework Design}
\label{sec:design}

The {\important device driver framework} (DDF) design contains
three primary components: an interprocess communication (IPC)
library that provides common methods for DDF communication, a
{\important device driver manager} (DDM) that monitors device drivers
registered with it, and a keyboard snooping driver that tests the correctness
of the framework.  \figurename~\ref{fig:ddf_layout} presents an overview of
the DDF. An in-depth discussion can be found in  the following subsections.

It is very important to realize that the DDF's primary purpose is not
to improve the raw performance (speed) of existing frameworks, but
merely to enhance the safety of driver execution and simplify the
driver implementor's task.  Performance is an important secondary
goal, so the DDF should either maintain or exceed current performance
levels seen in existing second-generation microkernel OS's.

	\begin{figure}[tbp]
	\begin{center}
	\mygraphic{ddf_layout.eps}
	\end{center}
	\caption{Device driver framework layout.}
	\label{fig:ddf_layout}
	\end{figure}

%\mychapter{Device Classes}
\mysubsection{Device Classes}

In our framework, there is no distinction amongst whether a driver is block, character, or raw;
% Originally, these three classes were going to exist, but 
%it was discovered
%during implementation that 
in the context of our framework, these classes are only relevant to higher level
subsystems (such as the file system) which do not affect driver operation.
In fact, drivers are each free to interact with the hardware in any way
they see fit.

%\mychapter{Device Scheduling}
\mysubsection{Device Scheduling}

Device drivers each have a first-in-first-out (FIFO) queue where requests to
use the device are sent.  This allows the drivers to pull new requests from
a single location one at a time while clients are allowed to add new
requests in parallel.

%In some monolithic OS, drivers themselves have direct
%access to this queue so that requests can be reordered to maximize performance;
%unfortunately, the current implementation of the DDF does not support this due
%to lack of capability in Minix 3 for it.  In fact, had drivers attempted
%to access multiple items in its queue using Minix 3's {\code receive()} call,
%the driver would have suffered a high likelihood of live-locking.

The framework is not designed to allow client processes to directly lock
drivers for exclusive use.  The reasons for this include avoiding a
performance penalty that would have occured by always having to check bits to
see if the calling client had appropriate access rights and because it was
intended that security be appropriately handled by other OS subsystems.  In
fact, the primary responsibility of this DDF is to provide ``utility'' to
driver writers and to leave security to dedicated subsystems.  An example of a
subsystem that could properly handle security for device driver access is a
device file system similar to FreeBSD's devfs\footnote{Provides a file
system-level interface for client processes to access device drivers.}.
This both simplifies the design and makes it more robust by concentrating on
one task only.

\mysection{User Presentation}

To the user, the DDM represents the drivers scattered in memory as a coherent
hierarchy.  In \figurename~\ref{fig:ddm_layout}, busses are internal nodes
while devices are leaves.

The DDM provides facilities for searching the tree in order to make finding
a driver or set of drivers easier.  For example, a client can ask the DDM
for a list of drivers attached to the PCI bus, and the DDM will return the
set of drivers claiming to belong to it.

All top-level independent drivers must attach to a special {\important null
bus} supported by the DDM.  The null bus provides a central access point into
the tree so that when a client wants a listing of all drivers on the
system, it can begin by enumerating all drivers on the null bus.  Mostly
bus drivers will be connected to the null bus, but other drivers that do not
directly depend on bus drivers for operation, such as a keyboard driver, can
also be found on the null bus.

	\begin{figure}[tbp]
	\begin{center}
	\scale{0.50}{\graphic{ddm_layout}}
	\end{center}
	\caption{Device driver manager's user presentation.}
	\label{fig:ddm_layout}
	\end{figure}


%\mychapter{Framework Abstraction}
\mysubsection{Framework Abstraction}

In this design, abstraction libraries are the core components needed to
enable communication among the operating system, device drivers,
hardware, DDM, and clients.  The libraries need to be
efficient; otherwise, the computational overhead could get in the
way of time-critical operations.  Therefore, a rich set of primitives that can
be used to pass raw data quickly are provided.  The libraries created for this
DDF are {\important libMem, libIO, libIPC, libDDM}, and {\important libDriver}, as discussed below.

The memory module, {\important libMem}, provides a method to copy large slabs
of memory from one process to another with the help of the native OS kernel.
However, it does not currently provide methods to (de)allocate memory on the
heap, because most modern compilers can now dynamically handle this on the
stack.  Regardless, a future edition of the DDF could handle slab
allocation operations for drivers that wish to use the heap for DMA rather
than the stack.

Another basic library needed for all higher-level functions in the DDF
is the I/O library, {\important libIO}. This library provides drivers with the
ability to read and write various-sized data to computers' I/O ports.
This eliminates the need for driver writers to write architecture-dependent
assembly to perform the same task.

The interprocess communication library, {\important libIPC}, provides
methods for all low-level interprocess communication in the DDF.
It also acts as the basis for all higher-level communication protocols used
by the DDM and drivers.

Building on {\em libIPC} is the DDM proxy, {\important libDDM}.  This library
provides easy-to-use methods for communicating with the DDM.  Internally,
it transforms method calls into IPC messages that are transmitted with
libIPC, but externally, access to the DDM is transparent and feels just like a
local method invocation.  Without a library such as this, clients
(typically drivers) would have to understand the exact protocol the DDM
uses for communication; that would be an error-prone and mundane task
for developers.

The driver library, {\important libDriver}, is used by both drivers and driver
clients.  From the driver end, driver writers inherit the classes in libDriver
and override the operations they wish to handle.  The operations currently
supported are {\important read, write, open, close, scatter, gather, I/O
control, initialization}, and {\important finalization}.  Each driver method
receives all arguments passed to it in a libIPC message customized to that
particular method.

From a client's perspective, {\em libDriver} acts as a proxy, handling
communication with drivers similarly to the way libDDM handles
communication between clients and the DDM.  This means driver method
invocations are as simple as ``a local method call'' with well-known
arguments.

The framework's abstraction libraries are object-oriented, providing a modern
approach to development and allowing the framework to be adapted to other
platforms more quickly and easily.  Currently, the libraries are statically
linked to code; however, the DDF can be modified to provide these libraries
as shared objects.


\mysection{Device Driver Manager}

The device driver manager is a server task that manages
drivers and allows other tasks to find and gain access to them.
Due to the fact that the DDM is the core process in the DDF, the DDM is
resilient to failures of devices or their drivers.

\mysubsection{Bootstrap Protocol}

Bootstrapping is a recurring theme with microkernel-based operating systems;
however, it is doubly difficult with a device driver framework.
A true microkernel is the first program
loaded, knows nothing about devices, and must rely on external drivers to do
I/O.  However, since a true microkernel does not know enough about the
hardware to load drivers off of secondary storage, it becomes useless.

Luckily, Minix 3 comes with some utilities that make it easy to create a
bootable image consisting of the microkernel and some processes which can
include the DDM and a few drivers.  This entirely eliminated the need for
me to implement this myself, but it is quite possible that porting this DDF
to another platform will reinstate the need to write this protocol.

\mysubsection{Fault Tolerance}

Since all drivers reside in their own address space and are assumed to
reside on the same system, the DDM does not need extensive fault tolerance.
When a process unexpectedly dies and another process tries to communicate
with it, the process manager will tell the calling process why the call
fails.  Likewise, when a failure is recognized, a log message is generated,
and it is up to the administrator to resolve the problem.  However, since
the existing implementation of this design is on Minix 3, Minix's
reincarnation server (RS) can reincarnate processes that have died
unexpectedly.

