\mychapter{Building Blocks of the Proposed Framework}

Our design for a user-space device driver framework is built upon work seen in
deva/fabrica~\cite{hurdl4ref2003} and is implemented on Minix 3.0.
%
%In the
%following subsections is information regarding deva/fabrica and Minix 3.0's
%existing driver framework.
%
Deva/fabrica is a user-space device driver framework implemented
for the GNU Hurd-L4 microkernel.	
% by Peter De Schriver and Daniel Wagner~\cite{hurdl4ref2003}.
It makes many practical decisions but was never completed and
suffers from some design complexities.  The goals of this {\em device driver framework} (DDF) are to be
comparable in speed to existing monolithic systems, portable, flexible,
consistent, tolerant of critical device driver
failures, and to provide a convenient interface.  To help meet the goals set
forth by the deva/fabrica project, our DDF needs to consider three primary components are: bus drivers, device drivers, and services. 

In \figurename~\ref{fig:deva_layout}, it can be seen that deva/fabrica models
the physical hardware layout of a system.  For example, there is one entry
point
%\footnote{Initial access point that provides information about how
%to get to other access points.}
, Bus$_0$, to all primary busses.  Bus$_0$ is a
virtual bus driver
%\footnote{A bus driver that does not own an actual
%hardware bus.} 
with information about how to access all directly-connected
children busses.  At the lowest level of the driver hierarchy are device
drivers, which are directly attached to bus drivers.

	\begin{figure}[tb]
	\begin{center}
	\psfrag{Process_1}{Process$_1$}
	\psfrag{Process_2}{Process$_2$}
	\psfrag{Process_3}{Process$_3$}
	\psfrag{Device_1}{Device$_1$}
	\psfrag{Device_2}{Device$_2$}
	\psfrag{Device_3}{Device$_3$}
	\psfrag{Bus_0}{Bus$_0$}
	\psfrag{Bus_1}{Bus$_1$}
	\psfrag{Bus_2}{Bus$_2$}
	\psfrag{Bus_3}{Bus$_3$}
	\mygraphic{deva_layout.eps}
	\end{center}
	\caption{Deva/fabrica layout.}%~\cite{hurdl4ref2003}.}   XXX
	\label{fig:deva_layout}
	\end{figure}

\mysubsection{Drivers}

The bus drivers watch I/O busses for insertion and removal events and notify
plugin managers of these events, providing them with the necessary information
to load the correct device driver.  The bus drivers also provide metadata
about a device to its driver, such as the bus addresses to which its
attached, the interrupt request (IRQ) numbers, the DMA channel IDs, etc.
Also, bus drivers provide a set of communication primitives for use between
the drivers and devices, such as send/receive and memory (un)mapping.  Support
for rescanning busses to check for devices whose drivers are not loaded is also
supported~\cite{hurdl4ref2003}; this particular ability is especially useful
during bootstrapping when some devices are not loaded because their drivers
depend on other devices to be running first, such as printers depending on
parallel port drivers.  

All I/O operations that need use of memory buffers require
the user-space application provide the device driver with the memory
containing the I/O to be done.  This is in contrast to monolithic kernels
where certain devices use kernel-space memory buffers.  The advantages to
having user-space memory buffers is that it avoids a memory-to-memory
copy of data~\cite{hurdl4ref2003, mckusick1996, l4ref2005} and prevents
user memory from entering protected memory.

\mysubsection{Driver Classes}

In the Deva/fabrica specification, device drivers may be one of nine device classes, including
character, block, human input, packet switched network, circuit switched
network, framebuffer, streaming audio, streaming video, and solid state
storage~\cite{hurdl4ref2003}.  Each of these classes has a specific type of
interface for communicating with the I/O bus to which they're attached.  As
mentioned earlier, the bus drivers themselves will provide this interface to
the device drivers.

Deva/fabrica's nine device classes can be simplified to just
two classes: {\important character} and {\important block}.  Character
devices cover five of deva/fabrica's device classes: human input,
packet and circuit switching networks, and streaming audio and video.
All other aforementioned device classes fall under the block devices
category.

%While character devices are generally used to transfer data as a stream of
%bytes~\cite{mckusick1996}, some systems such as BSD have a special character
%device class called the {\important raw device}~\cite{mckusick1996}.
%This device class is used by applications that have intimate knowledge
%about the device they're using, such as file system manipulators and video
%applications.  This allows user-space applications to directly access the
%I/O device.  However, this class is not needed in deva/fabrica, because all
%device drivers afford users the ability to directly manipulate I/O devices
%or to manipulate them via standard interfaces (i.e. open, read, write,
%close, lock).

\mysubsection{Services}

In \figurename~\ref{fig:deva_layout}, the
{\important Wortel} subsystem is a system-central IRQ logic
server that runs in kernel-space, and its primary
task is to enable IRQ rerouting~\cite{hurdl4ref2003}.  When IRQs are
shared amongst multiple drivers, it is their responsibility to pass
IRQs to peers.
 
The {\important plugin manager} found in \figurename~\ref{fig:deva_layout} is a process where bus events are sent to ask
deva to (un)load drivers.  Each bus driver is associated with exactly one
of the potentially many running plugin managers.  Also, the plugin
manager maintains a list of currently loaded drivers and provides this
information upon request.

The {\important deva} module found in \figurename~\ref{fig:deva_layout}
supports interaction with the host operating
system.  Its primary functions are to provide support for thread creation
and deletion, memory (de)allocation, I/O port mapping, driver (un)loading,
and bootstrapping.

With {\important deva}, thread creation and deletion, memory (de)allocation,
and I/O port mapping are done by asking the host operating system to perform
these tasks on its behalf.  Thus, {\important deva} provides an operating
system abstraction so drivers do not have to be modified for each operating
system.

%Deva's Driver (un)loading is used by the {\em deva} plugin managers
during the handling of bus events.  It is also used by user-space
processes that wish to manually load drivers.  The reason why {\em
deva} loads drivers instead of having the plugin manager so is simply
due to the fact that {\em deva} is asking the host operating system's
file system for a stored driver to load.  This is more consistent with
using {\em deva} as an abstraction to the operating system.

%\mysubsection{Bootstrapping}
%
%{\important Wortel} is the part of deva/fabrica that launches the device
%driver framework at boot time.  It launches {\important deva}, which
%proceeds to unpack an archive of stored device drivers and servers.
%After unpacking, it loads the first {\important plugin manager} and passes
%control over to it.  Finally, the plugin manager loads the root bus driver,
%Bus$_0$.
%
%\mysubsection{Problems}
%
%As mentioned earlier, deva/fabrica does not need nine device classes; instead,
%they can be simplified to just two classes.  The fact that deva/fabrica has
%nine device classes introduces extra complexity for both the framework
%implementors and users of the framework.  For example, nine device classes
%implies nine unique interfaces; however, a proper design approach is to
%use one interface with nine different underlying implementations.
%
%Finally, deva/fabrica was never fully realized, in part due to its
%complexity.  Therefore, it remains a partially complete project and
%has received no performance evaluations.
%
%The purpose of the Minix 3.0 project by Jorrit N. Herder was to greatly
%reduce the size of the Minix kernel and move device drivers out of the
%kernel and into user-space~\cite{herder2005}.  His document is a mostly
%complete analysis of his work and includes benchmarks that provide an
%indication of how other user-space device drivers may perform in a true
%microkernel environment.
%
%Before Jorrit Herder began the move of drivers from kernel-space to
%user-space, he ran analyses on Minix 2.0.4 to see how well IPC
%between user-space and kernel-space performed.  His results indicated that
%Minix 2.0.4 could pass a basic IPC message at roughly three times the
%speed of a system call.  Further, he indicated that the IPC overhead will
%likely only be 0.5\% of the total CPU time used by device
%drivers~\cite{herder2005}.
%
%Jorrit Herder's solution to placing device drivers into user-space was to
%allow the kernel to act as a proxy for device drivers when they needed to
%perform I/O and IPC~\cite{herder2005}.  The reasoning behind this move is
%because user-space processes no longer get direct access to the hardware
%and other processes' address space, and the kernel provides these features.
%
%Often, device drivers need access to information that is generally contained
%inside Minix kernel data structures, such as the process table or list of
%free memory chunks.  Since the new user-space device drivers can no longer
%directly access kernel memory, Jorrit Herder implemented a set of system
%calls that allowed drivers to copy these structures into their own address
%space, allowing safe use of the data~\cite{herder2005}.
%
%One other important change Jorrit Herder had to make to Minix was the
%prevention of potential deadlock, since it could render a system unusable.
%To handle this, Jorrit Herder bolstered the Minix kernel's asynchronous message
%facilities seen in \figurename~\ref{fig:minix_async_msg} so that the kernel
%could send critical messages to less trusted
%user-space servers without blocking the kernel~\cite{herder2005}.
%
%\figurename~\ref{fig:minix_async_msg} shows the kernel passing asynchronous
%hardware interrupts to device drivers.  The kernel does this by checking
%if the driver that needs to answer the request is ready, giving it the
%message and scheduling it to run if it is.  If the driver is not ready or
%if a race condition\footnote{Blocking a critical kernel task to answer another
%hardware interrupt would be a {\important race condition}.} exists, the kernel
%stores the pending message in a queue~\cite{herder2005}.
%
%	\begin{figure}[tbp]
%	\begin{center}
%	\psfrag{Device_1}{Device$_1$}
%	\mygraphic{minix_async_msg.eps}
%	\end{center}
%	\caption{Asynchronous message handling.}
%	\label{fig:minix_async_msg}
%	\end{figure}
%
%It is difficult to say how successful Jorrit Herder's project was, because
%he did no post-mortem examination of the user-space device driver performance.
%However, if Jorrit's preliminary test results are any indication, then the
%results are probably satisfactory.
