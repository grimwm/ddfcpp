%\mychapter{Framework Abstraction}
\mysubsection{Framework Abstraction}

In this design, abstraction libraries are the core components needed to
enable communication among the operating system, device drivers,
hardware, DDM, and clients.  The libraries need to be
efficient; otherwise, the computational overhead could get in the
way of time-critical operations.  Therefore, a rich set of primitives that can
be used to pass raw data quickly are provided.  The libraries created for this
DDF are {\important libMem, libIO, libIPC, libDDM}, and {\important libDriver}, as discussed below.

The memory module, {\important libMem}, provides a method to copy large slabs
of memory from one process to another with the help of the native OS kernel.
However, it does not currently provide methods to (de)allocate memory on the
heap, because most modern compilers can now dynamically handle this on the
stack.  Regardless, a future edition of the DDF could handle slab
allocation operations for drivers that wish to use the heap for DMA rather
than the stack.

Another basic library needed for all higher-level functions in the DDF
is the I/O library, {\important libIO}. This library provides drivers with the
ability to read and write various-sized data to computers' I/O ports.
This eliminates the need for driver writers to write architecture-dependent
assembly to perform the same task.

The interprocess communication library, {\important libIPC}, provides
methods for all low-level interprocess communication in the DDF.
It also acts as the basis for all higher-level communication protocols used
by the DDM and drivers.

Building on {\em libIPC} is the DDM proxy, {\important libDDM}.  This library
provides easy-to-use methods for communicating with the DDM.  Internally,
it transforms method calls into IPC messages that are transmitted with
libIPC, but externally, access to the DDM is transparent and feels just like a
local method invocation.  Without a library such as this, clients
(typically drivers) would have to understand the exact protocol the DDM
uses for communication; that would be an error-prone and mundane task
for developers.

The driver library, {\important libDriver}, is used by both drivers and driver
clients.  From the driver end, driver writers inherit the classes in libDriver
and override the operations they wish to handle.  The operations currently
supported are {\important read, write, open, close, scatter, gather, I/O
control, initialization}, and {\important finalization}.  Each driver method
receives all arguments passed to it in a libIPC message customized to that
particular method.

From a client's perspective, {\em libDriver} acts as a proxy, handling
communication with drivers similarly to the way libDDM handles
communication between clients and the DDM.  This means driver method
invocations are as simple as ``a local method call'' with well-known
arguments.

The framework's abstraction libraries are object-oriented, providing a modern
approach to development and allowing the framework to be adapted to other
platforms more quickly and easily.  Currently, the libraries are statically
linked to code; however, the DDF can be modified to provide these libraries
as shared objects.
